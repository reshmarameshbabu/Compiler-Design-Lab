% Created 2020-01-13 Mon 15:25
\documentclass[11pt]{article}
\usepackage[utf8]{inputenc}
\usepackage[T1]{fontenc}
\usepackage{fixltx2e}
\usepackage{graphicx}
\usepackage{longtable}
\usepackage{float}
\usepackage{wrapfig}
\usepackage{rotating}
\usepackage[normalem]{ulem}
\usepackage{amsmath}
\usepackage{textcomp}
\usepackage{marvosym}
\usepackage{wasysym}
\usepackage{amssymb}
\usepackage{hyperref}
\tolerance=1000
\usepackage{palatino}
\usepackage[top=1in,bottom=1.25in,left=1.25in,right=1.25in]{geometry}
\usepackage{setspace}
\setcounter{secnumdepth}{1}
\author{Praveen Kumar R}
\date{\today}
\title{Exercise 2: Lexical Analyser using Lex tool}
\hypersetup{
  pdfkeywords={},
  pdfsubject={},
  pdfcreator={Emacs 25.2.2 (Org mode 8.2.10)}}
\begin{document}

\maketitle
\begin{export}
\linespread{1.2}
\end{export}
\begin{center}
\begin{tabular}{lr}
Assignment & 2\\
Reg No & 312217104114\\
Name & Praveen Kumar R\\
\end{tabular}
\end{center}


\section{Lex Program}
\label{sec-1}
\begin{verbatim}
%{
#include<stdio.h>
#include<string.h>
#include<stdlib.h>
typedef struct table {
    char identifier[32];
    char type[5];
    int start;
    int size;
    double value;
} Table;
Table t[100];
int t_index = 0,base = 1000,flag = 0,fg[20],cn = 0; 
char symbols[20][20],values[20][20];
%}
keyword ("auto"|"break"|"case"|"char"|"const"|"continue"|"default"|
"do"|"double"|"else"|"enum"|"extern"|"float"|"for"|"goto"|"if"|
"int"|"long"|"register"|"return"|"short"|"signed"|"sizeof"|
"static"|"struct"|"switch"|"typedef"|"union"|"unsigned"|"void"|
"volatile"|"while")
function [a-zA-Z_][a-zA-Z0-9_]*[(].*[)]
identifier [a-zA-Z_][a-zA-Z0-9_]*
int_constant [0-9]+
float_constant [0-9]+.[0-9]+
%%
^#.* { printf("%s - preprocessor directive\n", yytext); }
{keyword} {
    int i = 0;
    char s[10]; strcpy(s, yytext);
    while(s[i++] != '\0') if(s[i] == ' ' ||
 s[i] == '\t' || s[i] == '\n') s[i] = ' ';
    printf("%s - keyword\n", s);
    if(strcmp(s,"int")==0) flag = 2;
    else if(strcmp(s,"float")==0) flag  = 4;
}
{function} { printf("%s - function call\n", yytext); }
{identifier} { printf("%s - identifier\n", yytext);
                strcpy(symbols[cn],yytext); }
{int_constant} { printf("%s - integer constant\n", yytext); 
                strcpy(values[cn],yytext);
                fg[cn] = flag;
                cn++;
                }
{float_constant} { printf("%s - float/double constant\n", yytext);
                strcpy(values[cn],yytext);
                fg[cn] = flag;
                cn++; }
("<"|"<="|">"|">="|"=="|"!=") { printf("%s - relational operator\n", yytext); }
= { printf("%s - assignment operator\n", yytext); }

[{}(),;] { printf("%s - special character\n", yytext); }
. { }
\n { }
%%
int main(int argc, char* argv[])
{
    yyin = fopen(argv[1], "r");
    yylex();
    printf("SYMBOL TABLE\nTYPE\tSYMBOL\tSIZE\tADDRESS\tVALUE\n");
    for (int i = 0; i<cn ;i++){
        printf("%s\t %s\t %d\t %d\t %s\n",fg[i]==2?"int":"float",
symbols[i],fg[i],base,values[i]);
        base = base+ fg[i];
    }
    return 0;
}
\end{verbatim}

\section{Output}
\label{sec-2}
\begin{verbatim}
#include<stdio.h> - preprocessor directive
main() - function call
{ - special character
int - keyword
a - identifier
= - assignment operator
10 - integer constant
, - special character
b - identifier
= - assignment operator
20 - integer constant
; - special character
float - keyword
c - identifier
= - assignment operator
10.4 - float/double constant
, - special character
d - identifier
= - assignment operator
20.5 - float/double constant
; - special character
if - keyword
( - special character
a - identifier
> - relational operator
b - identifier
) - special character
printf("a is greater") - function call
; - special character
else - keyword
printf("b is greater") - function call
; - special character
} - special character
SYMBOL TABLE
TYPE	SYMBOL	SIZE	ADDRESS	VALUE
int	 a	 2	 1000	 10
int	 b	 2	 1002	 20
float	 c	 4	 1004	 10.4
float	 d	 4	 1008	 20.5
\end{verbatim}
% Emacs 25.2.2 (Org mode 8.2.10)
\end{document}
